\documentclass{article}
\usepackage[margin=1in]{geometry}
\usepackage{setspace}
\usepackage{amsmath}
\usepackage{amssymb}

\title{Math 115A Midterm 2}
\author{Jiaping Zeng}
\date{5/16/2020}

\begin{document}
\setstretch{1.35}
\begin{itemize}
    \item [1.]
          \begin{itemize}
              \item [(a)] To be proved: $T$ is an isomorphism $\Leftrightarrow$ $T(B)$ is a basis for $W$.
              \begin{itemize}
                  \item [$\Rightarrow:$] Since $T$ is injective and $B$ is linearly independent, $T(B)$ is also linearly independent. Let $\text{dim}(V)=n$, then $B$ has $n$ elements by definition of dimension. Since $T$ is an isomorphism, $\text{dim}(V)=\text{dim}(W)=n$. Then, $T(B)$ is a set of $n$ linearly independent elements in the $n$-dimensional space $W$ which means that it is indeed a basis of $W$.
                  \item [$\Leftarrow:$] Again let $\text{dim(V)}=n$, then both $B$ and $T(B)$ have exactly $n$ elements. Since $T(B)$ is a basis of $W$, $\text{dim}(W)=n=\text{rank}(T)$ and therefore $T$ is surjective. Then, by rank-nullity theorem, $\text{nullity}(T)=\text{dim}(W)-\text{rank}(T)=0$ which implies that $T$ is also injective. Thus $T$ is an isomorphism.
              \end{itemize}
              \item [(b)] A diagonalisable isomorphism $T$ means that $[T]_\beta$ is a diagonal matrix for some basis $\beta$ with a nonzero nullity. For example, let $T:\mathbb{R}_2[x]\rightarrow\mathbb{R}_2[x]$ such that $T(p)=x\frac{dp}{dx}$. In addition, let $\beta=\{1,x,x^2\}$ be an ordered basis of $\mathbb{R}_2[x]$. Then, 
              \[[T]_\beta^\beta=\begin{pmatrix}0&0&0\\0&1&0\\0&0&2\end{pmatrix}\] which we can see is indeed diagonal. Thus, $T$ is diagonalisable. At the same time, $T$ is not injective, and therefore not an isomorphism, as its non-empty kernel contains the set of all constants.
          \end{itemize}
\end{itemize}
\newpage
\begin{itemize}
    \item [2.] Since vectors in $V$ has the form $\begin{pmatrix}a&b\\c&-a\end{pmatrix}$, $\beta=\{\begin{pmatrix}1&0\\0&-1\end{pmatrix},\begin{pmatrix}0&1\\0&0\end{pmatrix},\begin{pmatrix}0&0\\1&0\end{pmatrix}\}$ is a basis of $V$. We can then apply $T$ to each element of $\beta$ to find $[T]_\beta^\beta$: \[T(\begin{pmatrix}1&0\\0&-1\end{pmatrix})=\begin{pmatrix}0&0\\0&0\end{pmatrix}\]\[T(\begin{pmatrix}0&1\\0&0\end{pmatrix})=\begin{pmatrix}0&2\\0&0\end{pmatrix}\]\[T(\begin{pmatrix}0&0\\1&0\end{pmatrix})=\begin{pmatrix}0&0\\-2&0\end{pmatrix}\] Then, \[[T]_\beta^\beta=\begin{pmatrix}0&0&0\\0&2&0\\0&0&-2\end{pmatrix}.\]
          \begin{itemize}
              \item [(a)] The characteristic polynomial can be found as follows: $P_T(\lambda)=\text{det}(T-\lambda I)=-\lambda(2-\lambda)(-2-\lambda)=-\lambda^3+4\lambda$. Then, we can find the eigenvalues by solving $P_T(\lambda)=0$, which results in $\lambda=-2,0,2$.
              \item [(b)] We can find the eigenvectors by solving $(T-\lambda I)v=0$ for each $\lambda$.\\$\lambda_1=-2$:\[\begin{pmatrix}2&0&0\\0&4&0\\0&0&0\end{pmatrix}\begin{pmatrix}a\\b\\c\end{pmatrix}=0\]\[\implies a=b=0\]Then $v_1=\begin{pmatrix}0\\0\\1\end{pmatrix}$ is an eigenvector, which corresponds to the matrix $\begin{pmatrix}0&0\\1&0\end{pmatrix}$.\\$\lambda_2=0$:\[\begin{pmatrix}0&0&0\\0&2&0\\0&0&-2\end{pmatrix}\begin{pmatrix}a\\b\\c\end{pmatrix}=0\]\[\implies b=c=0\]Then $v_2=\begin{pmatrix}1\\0\\0\end{pmatrix}$ is an eigenvector, which corresponds to the matrix $\begin{pmatrix}1&0\\0&-1\end{pmatrix}$.\\$\lambda_3=2$:\[\begin{pmatrix}-2&0&0\\0&0&0\\0&0&-4\end{pmatrix}\begin{pmatrix}a\\b\\c\end{pmatrix}=0\]\[\implies a=c=0\]Then $v_3=\begin{pmatrix}0\\1\\0\end{pmatrix}$ is an eigenvector, which corresponds to the matrix $\begin{pmatrix}0&1\\0&0\end{pmatrix}$.
              \item [(c)] Since $[T]_\beta^\beta$ is a diagonal matrix as shown above, $T$ is diagonalisable.
          \end{itemize}
\end{itemize}
\newpage
\begin{itemize}
    \item [3.]
          \begin{itemize}
              \item [(a)] If $0$ is an eigenvalue of $T$, we know that its corresponding eigenspace $E_0\neq\{0\}$. By definition of eigenspace, $E_0=\text{ker}(T-0I)=\text{ker}(T)$. Therefore $\text{ker}(T)\neq\{0\}$ which means that $T$ is not injective and therefore not an isomorphism.
              \item [(b)] If $0$ is not an eigenvalue of $T$, $\text{det}(T-0I)=\text{det(T)}\neq 0$. Then, $T$ is full rank, i.e. $\text{rank}(T)=\text{dim}(V)$ and $T$ is surjective. Since $T$ is a map from $V$ to itself, $\text{nullity(T)}=\text{dim}(V)-\text{rank}(T)=0$ by rank-nullity theorem and $T$ is injective. Therefore $T$ is indeed an isomorphism.
          \end{itemize}
\end{itemize}
\end{document}