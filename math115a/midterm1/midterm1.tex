\documentclass{article}
\usepackage[margin=1in]{geometry}
\usepackage{setspace}
\usepackage{amsmath}
\usepackage{amssymb}

\title{Math 115A Midterm 1}
\author{Jiaping Zeng}
\date{4/22/2020}

\begin{document}
\setstretch{1.35}
\begin{itemize}
	\item [1.]
	      \begin{itemize}
		      \item [(a)] $V=\text{Mat}_{2x2}(\mathbb{C})=\{\begin{pmatrix}a&b\\c&d\end{pmatrix} \mid a,b,c,d \in \mathbb{C}\}$
		      \item [(b)] $B=\{\begin{pmatrix}1&0\\0&0\end{pmatrix}, \begin{pmatrix}0&1\\0&0\end{pmatrix}, \begin{pmatrix}0&0\\1&0\end{pmatrix}, \begin{pmatrix}0&0\\0&1\end{pmatrix}\}$
		      \item [(c)] $W=\{\begin{pmatrix}a&a\\b&b\end{pmatrix} \mid a,b \in \mathbb{C}\}$ with basis $\{\begin{pmatrix}1&1\\0&0\end{pmatrix}, \begin{pmatrix}0&0\\1&1\end{pmatrix}\}$
		      \item [(d)] $C=\{\begin{pmatrix}1&1\\0&0\end{pmatrix}, \begin{pmatrix}0&0\\1&1\end{pmatrix}, \begin{pmatrix}1&0\\0&0\end{pmatrix}, \begin{pmatrix}0&0\\0&1\end{pmatrix}\}$
	      \end{itemize}
\end{itemize}
\newpage
\begin{itemize}
	\item [2.] Let $p(x)=a_0+a_1x+...+a_nx^n$, then $p(1)=2p(0)$ implies that $a_0+...+a_n=2a_0$. Subtract $a_0$ from both sides and we have $a_0=a_1+...+a_n$.
	      \begin{itemize}
		      \item [(a)] Per Theorem 1.3, we need to verify the following three conditions:
		            \begin{itemize}
			            \item [(1)] $0 \in U_n$: The zero in $\mathbb{R}_n[x]$ is $p_0(x)=0$. By evaluating $p_0(x)=0$ at $1$ and $0$, we can easily see that $p_0(1)=2p_0(0)=0$. Therefore, $0 \in U_n$.
			            \item [(2)] $x+y \in U_n \text{ for } x,y \in U_n$: Let $u,v \in U_n$ such that $u=a_0+a_1x+...+a_nx^n$ and $v=b_0+b_1x+...+b_nx^n$. Then $u+v=(a_0+b_0)+(a_1+b_1)x+...+(a_n+b_n)x^n$. As shown above, $a_0=a_1+...+a_n$ and $b_0=b_1+...+b_n$. Therefore, $a_0+b_0=(a_1+b_1)+...(a_n+b_n)$, which means that $u+v \in U_n$.
			            \item [(3)] $cx \in W \text{ for } c \in \mathbb{F}, x \in U_n$: Using $u$ defined in the previous part, $cu=ca_0+ca_1x+...+ca_nx^n$. We can verify $cu \in U_n$ by checking that $ca_0=ca_1+...+ca_n$, which is indeed true upon multiplying both sides by $c^{-1}$. Therefore, $cu \in U_n$.
		            \end{itemize}
		      \item [(b)] To verify that $B$ is indeed a basis for $U_n$, we need to show that $B$ is both generating and linearly independent. By definition, $B=\{1+x^a \mid 1 \leq a \leq n\}=\{1+x, ..., 1+x^n\}$. We can see that the vectors in $B$ are indeed linearly independent by comparing coefficients. To show that $B$ is generating, take a generic $u \in U_n$ such that $u=a_0+a_1x+...+a_nx^n$. By definition, we also have $a_0=a_1+...+a_n$. Then, we can write $u$ as a linear combination of the vectors in $B$ as follows: $u=a_1(1+x)+...+a_n(1+x^n)=(a_1+...+a_n)+a_1x+...+a_nx^n=a_0+a_1x+...+a_nx^n$. Therefore, $B$ is a basis for $U_n$.
		      \item [(c)] To show that $T(cu+v)=cT(u)+T(v)$, take $u,v \in U_n$ such that $u=a_0+a_1x+...+a_nx^n$ and $v=b_0+b_1x+...+b_nx^n$. Then, $T(cu+v)=(ca_0+b_0)+(ca_1+b_0)x+...+(ca_n+b_n)x^n$, which is indeed equal to $cT(u)+T(v)=c(a_0+a_1x+...+a_nx^n)+(b_0+b_1x+...+b_nx^n)$.
	      \end{itemize}
\end{itemize}
\newpage
\begin{itemize}
	\item [3.] To prove that $\#B=\text{dim }V \Leftrightarrow U=V$, we need to prove both directions:
	      \begin{itemize}
		      \item [$\Rightarrow$:] If $\#B=\text{dim }V$, then $\#B=\text{dim }V=\text{dim }U$ since $B$ is a basis of $U$. In addition, $B$ is a linearly independent set by definition of basis. Therefore we have a linearly independent set $B$, with size of $\text{dim }V$. Using the Replacement Theorem, there exists a subset $H$ of size $(\text{dim }V-\text{dim }V)=0$ such that $B \cup H$ generates V, i.e. $H=\emptyset$ and $B$ generates V. Then, $B$ is a basis for both $U$ and $V$, which means that $U=V$.
		      \item [$\Leftarrow$:] Since $B$ is a basis of U, by definition of dimension, $\#B=\text{dim } U$. Then, since $U=V$, $\text{dim }U=\text{dim }V=\#B$.
	      \end{itemize}
\end{itemize}
\end{document}