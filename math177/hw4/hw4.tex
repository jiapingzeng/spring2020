\documentclass{article}
\usepackage[margin=1in]{geometry}
\usepackage{actuarialangle}
\usepackage{setspace}
\usepackage{amsmath}
\usepackage{relsize}

\title{Math 177 Homework 4}
\date{5/10/2020}
\author{Jiaping Zeng}

\begin{document}
\setstretch{1.35}
\maketitle

\section*{Section 4.1}
\begin{itemize}
	\item [1.] Since yield rate $7.7\%>7.2\%$, $(b)>(a)$ and $(d)>(c)$. In addition, since the yield rate is greater than the coupon rate for all options, $(a)>(c)$ and $(b)>(d)$. Therefore, $(b)>(a)>(d)>(c)$.
	\item [2.] $115.84=Cv_{3.5\%}^{24}+100*3.5\%a_{\angl{24}3\%} \implies \boxed{C=114.99}$
	\item [3.] $5083.49(1+j)^{20}=10000 \implies j=0.0344081$; $X=10000v_j^{20}+10000*10\%a_{\angl{20}j}=\boxed{12227.90}$
	\item [4.] Let $n$ be the number of half-years. $Fv_{2.5\%}^{16}+F*3\%a_{\angl{16}2.5\%}=Fv_{2.5\%}^n+F*2.75\%a_{\angl{n}2.5\%} \implies n=42.84 \implies \boxed{21.42\text{ years}}$
	\item [6.] Let $j$ be the quarterly yield rate and $i$ the nominal annual yield rate. Then, $800=1000v_j^{100}+1000*2.5\%a_{\angl{100}j} \implies j=0.0316179 \implies \boxed{i=0.126417}$
	\item [7.] $L=P=1000v_{4\%}^{20}+1000*5\%a_{\angl{20}4\%}=1135.903263$\\$\text{Net gain}=50s_{\angl{20}3\%}+1000-L(1+7\%)^{10}=\boxed{109.03}$
	\item [11.]
	      \begin{itemize}
		      \item [I.] False since $i_2>i_1 \implies K_2<K_1 \implies P_2<P_1$.
		      \item [II.] True since $i_2>i_1 \implies v_{i_2}<v_{i_1} \implies r_2a_{\angl{n}i_2}>r_1a_{\angl{n}i_1}$.
		      \item [III.] False. $i_2>i_1 \implies v_{i_2}<v_{i_1} \implies PV_A > PV_B$
	      \end{itemize}
	\item [12.] Let prices of the bonds be $X$ and $Y$ with coupon rates $2r$ and $r$ respectively.\\$X=100+100(2r-1.5\%)a_{\angl{n}3\%}$; $Y=100+100(r-1.5\%)a_{\angl{3\%}n}$\\$X+Y=240 \text{ and } X-Y=24 \implies X=132, Y=108$\\$\implies n=13.05 \text{ and } r=0.0225$\\Therefore the coupon rates are $2.25\%$ and $4.50\%$ respectively.
	\item [15.] $P=1000v_{5\%}^{40}+1000*4\%a_{\angl{40}5\%}=828.409137$\\$P=Cv_{5\%}^{20}+1000*4\%a_{\angl{20}5\%} \implies \boxed{C=875.38}$
	\item [18.] $g=\frac{Fr}{C} \implies F=\frac{Cg}{r}$, $r=\frac{Cg}{F}$, $Fr=Cg$, then \begin{itemize}
		      \item [(4.2E)] $P=Cv_j^n+Fra_{\angl{n}j}=Cv_j^n+Cga_{\angl{n}j}$
		      \item [(4.3E)] $P=C+(Fr-Cj)a_{\angl{n}j}=P=C+(Cg-Cj)a_{\angl{n}j}=C+C(g-j)a_{\angl{n}j}$
		      \item [(4.4E)] $P=K+\frac{r}{j}(F-K)=K+\frac{Fr}{j}-\frac{r}{j}K=K+\frac{Cg}{j}-\frac{Cg}{Fj}K=K+\frac{g}{j}(C-K)$
	      \end{itemize}
\end{itemize}

\section*{Section 4.2}
\begin{itemize}
	\item [1.] Total amount paid: $F+Frn$\\Total interest repaid: $Frn-F(r-j)a_{\angl{n}j}$\\Total priciple repaid: $F+F(r-j)a_{\angl{n}j}$
	\item [4.] $B_{t+1}=B_t(1+j)-K_{t+1}=90(1+3.3\%)-100*2.5\%=\boxed{90.47}$
	\item [5.] $100(r-3.5\%)=1.00 \implies r=0.045$; $136=100+100(r-3.5\%)a_{\angl{n}3.5\%} \implies \boxed{n=26}$
\end{itemize}

\section*{Section 5.1}
\begin{itemize}
	\item [4.] $\sum_{n=0}^3 C_n^Av^n=0 \implies -5+3.72v+4v^3=0 \implies \boxed{j_A=0.253304}$\\$\sum_{n=0}^3 C_n^Bv^n=0 \implies -5+3v+1.7v^2+3v^3=0 \implies \boxed{j_B=0.253280}$\\We can set the two transactions equal: $-5+3.72v+4v^3=-5+3v+1.7v^2+3v^3 \implies i=0.111111 \text{ or } i=0.25$. Then by substituting values of $i$ we can see that $B>A$ for $0.111111<i<0.25$ and $A>B$ otherwise.
	\item [7.]
	      \begin{itemize}
		      \item [(a)] Since $C_n<0$ for $0\leq n\leq 23$ and $C_{24}>0$, there exists a unique $i>-1$.
		      \item [(b)] Since $i$ is unique, $F_{24}>0 \implies Y-150000-24(10000)-2(10000)-740000>0 \implies \boxed{Y>938800}$
	      \end{itemize}
	\item [9.]
	      \begin{itemize}
		      \item [(a)] Net profit: $1000000(1+i)^{15}+(950000-5*10000\ddot{s}_{\angl{1}4\%})(1+i)^{14}+(910000-4*10000\ddot{s}_{\angl{2}4\%})(1+i)^{13}+(870000-4*10000\ddot{s}_{\angl{3}4\%})(1+i)^{12}+(840000-3*10000\ddot{s}_{\angl{4}4\%})(1+i)^{11}+(910000-3*10000\ddot{s}_{\angl{5}4\%})(1+i)^{10}+(910000-2*10000\ddot{s}_{\angl{6}4\%})(1+i)^{9}+(910000-2*10000\ddot{s}_{\angl{7}4\%})(1+i)^{8}+(910000-4*10000\ddot{s}_{\angl{8}2\%})(1+i)^{7}+(910000-10000\ddot{s}_{\angl{9}4\%})(1+i)^{6}+(910000-10000\ddot{s}_{\angl{10}4\%})(1+i)^{5}+(910000-10000\ddot{s}_{\angl{11}4\%})(1+i)^{4}+(910000-10000\ddot{s}_{\angl{12}4\%})(1+i)^{3}+(910000-10000\ddot{s}_{\angl{13}4\%})(1+i)^{2}+(910000-10000\ddot{s}_{\angl{14}4\%})(1+i)^{1}-69*300000$
		      \item [(b)] We can find the value of $i$ by setting the above expression to equal to 0 and solve for $i$.
	      \end{itemize}
	\item [11.] $-1000000-\int_0^5 200000e^{-\delta t}dt+\int_1^3 250000(1+t)e^{-\delta t}dt+\int_3^5 400000(5.5-t)e^{-\delta t}dt=0 \implies \boxed{\delta=0.371795}$
\end{itemize}

\section*{Section 5.2}
\begin{itemize}
	\item [1.] $(1+X)^2=\frac{1310000+250000}{1000000}*\frac{1265000+150000}{1310000}*\frac{1540000+250000}{1265000}*\frac{1420000+150000}{1540000} \implies \boxed{X=0.0913523}$
	\item [2.] $1+0\%=\frac{12}{10}*\frac{X}{12+X}\implies\boxed{X=60}$\\$10(1+Y)+60(1+\frac{6}{12}Y)=60 \implies \boxed{Y=-0.25}$
	\item [3.] Fund after 1 year (dollar-weighted): $100000(1+x)-8000(1+\frac{9}{12}x)$\\Time-weighted: $1+x=\frac{103992}{100000(1+x)-8000(1+\frac{9}{12}x)}$\\$\implies \boxed{x=0.0624991}$
	\item [4.] 6 months: $1+Y=\frac{40}{50}*\frac{80}{40+20}*\frac{157.50}{80+80} \implies Y=0.05 \implies i=(1+X)^2-1=0.1025$\\12 months: $1.1025=\frac{40}{50}*\frac{80}{40+20}*\frac{175}{80+80}*\frac{X}{175+75} \implies \boxed{X=236.25}$
	\item [6.] Account K: $100(1+i)-X(1+\frac{6}{12}i)+2X(1+\frac{3}{12}i)=125$\\Account L: $1+i=\frac{125}{100}*\frac{105.8}{125-X}$\\$\implies X=10\text{, }\boxed{i=0.15}$
\end{itemize}

\section*{Section 5.3}
\begin{itemize}
	\item [2.] $i=\dfrac{2I}{2F(t_1)+N}=\dfrac{2I}{2F(t_1)+(F(t_2)-F(t_1)+I)}=\dfrac{2I}{F(t_1)+F(t_2)-I}$
	\item [3.] $16147=-10000*(1+15\%)^t*(1+9\%)^{10-t}+Xs_{\angl{t}15\%}*(1+9\%)^{10-t}+Xs_{\angl{10-t}9\%} \implies \boxed{t=6}$
\end{itemize}

\end{document}