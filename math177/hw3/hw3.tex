\documentclass{article}
\usepackage[margin=1in]{geometry}
\usepackage{actuarialangle}
\usepackage{setspace}
\usepackage{amsmath}
\usepackage{relsize}

\title{Math 177 Homework 3}
\date{5/5/2020}
\author{Jiaping Zeng}

\begin{document}
\setstretch{1.35}
\maketitle

\section*{Section 3.1}
\begin{itemize}
	\item [1.]
	      \begin{itemize}
		      \item [(i)] $L=1000a_{\angl{5}10\%}+500v^5a_{\angl{5}10\%}=\boxed{4967.68}$
		      \item [(ii)] $B_3=L(1+10\%)^3-1000s_{\angl{3}10\%}=\boxed{3301.98}$
		      \item [(iii)] $I_4=B_3*10\%=\boxed{330.20}$\\$P_4=1000-I_4=\boxed{669.80}$
		      \item [(iv)] $B_8=500_{\angl{2}10\%}=\boxed{867.77}$
	      \end{itemize}
	\item [2.] $v=\frac{1}{1+\frac{9\%}{12}}$; $B_{40}=\sum_{n=1}^{20} 1000v^n(1-2\%)^{39+n}=\boxed{6889.11}$
	\item [4.]
	      \begin{itemize}
		      \item [(i)] Let $X$ be the monthly payment amount. Then $20000=X(12+a_{\angl{36}\frac{6\%}{12}}) \implies \boxed{X=445.72}$\\$B_1=20000-12X=\boxed{14651.36}$
		      \item [(ii)] $20000=X(a_{\angl{12}\frac{3\%}{12}}+v_{\frac{3\%}{12}}^{12}a_{\angl{36}\frac{5\%}{12}}) \implies \boxed{X=452.61}$\\$B_1=20000(1+\frac{3\%}{12})^{12}-Xs_{\angl{12}\frac{3\%}{12}}=\boxed{15101.68}$
	      \end{itemize}
	\item [5.] Bank Y monthly interest: $(1+i)^6=1+\frac{14\%}{2} \implies i=0.0113403$; $P_t=\frac{19800}{36}=550 \implies I_t=iB_{t-1}=\frac{12\%}{12}[19800-(t-1)P_t]\implies Price=P_ta_{\angl{20}i}+\sum_{n=16}^{36}(v_i^{n-16}I_n)=\boxed{10857.27}$
	\item [6.] $L=\sum_{i=1}^nP_n=\sum_{i=1}^n(K_n-I_n)=K_T-I_T$
	\item [9.] $B_{10}=L=1000$\\$B_{20}=B_{10}(1-5\%)^{10}=598.74$\\$Xa_{\angl{10}10\%}=B_{20} \implies \boxed{X=97.44}$
	\item [11.]
	      \begin{itemize}
		      \item [(a)] $1000=K(2a_{\angl{144}\frac{12\%}{12}}-a_{\angl{72}\frac{12\%}{12}})\implies \boxed{K=9.89}$\\
		            \begin{tabular}{|c|c|c|c|c|}
			            \hline
			            $t$  & $K_t$  & $I_t$   & $P_t$   & $B_t$     \\
			            \hline
			            $0$  & $-$    & $-$     & $-$     & $1000$    \\
			            \hline
			            $1$  & $9.89$ & $10.00$ & $-0.11$ & $1000.11$ \\
			            \hline
			            $2$  & $9.89$ & $10.00$ & $-0.11$ & $1000.22$ \\
			            \hline
			            $3$  & $9.89$ & $10.00$ & $-0.11$ & $1000.33$ \\
			            \hline
			            $4$  & $9.89$ & $10.00$ & $-0.11$ & $1000.44$ \\
			            \hline
			            $5$  & $9.89$ & $10.00$ & $-0.11$ & $1000.55$ \\
			            \hline
			            $6$  & $9.89$ & $10.01$ & $-0.12$ & $1000.67$ \\
			            \hline
			            $7$  & $9.89$ & $10.01$ & $-0.12$ & $1000.79$ \\
			            \hline
			            $8$  & $9.89$ & $10.01$ & $-0.12$ & $1000.91$ \\
			            \hline
			            $9$  & $9.89$ & $10.01$ & $-0.12$ & $1001.03$ \\
			            \hline
			            $10$ & $9.89$ & $10.01$ & $-0.12$ & $1001.15$ \\
			            \hline
			            $11$ & $9.89$ & $10.01$ & $-0.12$ & $1001.27$ \\
			            \hline
			            $12$ & $9.89$ & $10.01$ & $-0.12$ & $1001.39$ \\
			            \hline
		            \end{tabular}
		      \item [(b)] Using the table above, we can see that $P_{t+1}$ is indeed equal to $P_t(1+i)+K_{t+1}-K_t$ for $1\leq t\leq 12d$.
		      \item [(c)] $B_{72}=1000+0.11s_{\angl{72}1\%}=1011.52$\\$B_{144}=1011.52-(2K-iB_{72})s_{\angl{72}1\%}=\boxed{0.02}$
	      \end{itemize}
	\item [13.]
	      \begin{itemize}
		      \item [(a)] $P_6=K_6-I_6=500-(1000a_{\angl{10}i}-a_{\angl{1}i})i=500(1-2a_{\angl{10}i}i+vi)=500[-2(1-v^{10})-(1-vi)]=500(2v^{10}-v)$
		      \item [(b)] $P_6=K_6-I_6=500-(1000a_{\angl{10}i}-a_{\angl{1}i})i=500(1-2a_{\angl{10}i}i+vi)=500[1-i(2a_{\angl{10}i}-v)]$
		      \item [(c)] $P_6=P_1(1+i)^5=(500-Li)(1+i)^5$
	      \end{itemize}
\end{itemize}

\section*{Section 3.2}
\begin{itemize}
	\item [1.] $B_t=L(1+t)^t-Ks_{\angl{t}i}=Ka_{\angl{n}i}(1+i)^t-Ks_{\angl{t}i}=(Ks_{\angl{t}i}+Ka_{\angl{n-t}i})-Ks_{\angl{t}i}=Ka_{\angl{n-t}i}$\\$B_t=L(1+t)^t-Ks_{\angl{t}i}=L+Lis_{\angl{t}i}-Ks_{\angl{t}i}=L+s_{\angl{t}i}(Li-K)=L+P_1s_{\angl{t}i}$
	\item [4.] Let $X$ be the monthly payment amount. Then $L=Xa_{\angl{60}0.01}$ and $B_t=Xa_{\angl{60-t}0.01}$. $B_t=\frac{L}{2} \implies \boxed{t=34.41}$.
	\item [7.] $X(1+6\%)^{10}-X=\frac{10X}{a_{\angl{10}6\%}}-X+356.54 \implies \boxed{X=825.00}$
	\item [8.] Under option (i), let $X$ be the annual payment amount. Then, $Xa_{\angl{10}8.07\%}=2000 \implies X=299.00$. Then, under option (ii), $\sum_{n=0}^9 200+200(10-n)i)=10X \implies 11000i+2000=2990 \implies \boxed{i=0.09}$
	\item [11.] Let $t-1,t,t+1$ be the dates of the given consecutive payments. Then, $P_t=B_t-B_{t-1}=5190.72-5084.68=106.04$ and $P_{t+1}=B_{t+1}-B_t=5084.68-4973.66=111.02$. Then, $i=\frac{P_{t+1}}{P_t}-1=0.0469634 \implies I_t=iB_{t-1}=243.77 \implies K=P_t+I_t=\boxed{349.81}$
	\item [13.] Scheme (i): Let each payment be $X$. Then, $L=Xa_{\angl{n}i} \implies X=\frac{L}{a_{\angl{n}i}}$; $I=nX-L=\boxed{L(na_{\angl{n}i}-1)}$\\Scheme (ii): $I=Li\sum_{t=0}^{n-1}\frac{n-t}{n}=\boxed{\frac{1}{2}Li(n+1)}$
	\item [14.] $A=125000a_{\angl{5}5\%}=\boxed{541184.58}$\\$B=75000a_{\angl{5}5\%}=\boxed{324710.75}$\\$C=10000(Da)_{\angl{5}5\%}=\boxed{134104.67}$\\$A+B+C=541184.58+324710.75+134104.67=1000000$
	\item [15.] As shown below, $n=10$ and $Payment=58.40$\\
	      \begin{tabular}{|c|c|c|c|c|}
		      \hline
		      $t$  & $K_t$  & $I_t$   & $P_t$   & $B_t$     \\
		      \hline
		      $0$  & $-$    & $-$     & $-$     & $1000.00$    \\
		      \hline
		      $1$  & $100$ & $10.00$ & $90.00$ & $910.00$ \\
		      \hline
		      $2$  & $100$ & $9.10$ & $90.90$ & $819.10$ \\
		      \hline
		      $3$  & $100$ & $8.19$ & $91.81$ & $727.29$ \\
		      \hline
		      $4$  & $100$ & $7.27$ & $92.73$ & $634.56$ \\
		      \hline
		      $5$  & $100$ & $6.35$ & $93.65$ & $540.91$ \\
		      \hline
		      $6$  & $100$ & $5.41$ & $94.59$ & $446.32$ \\
		      \hline
		      $7$  & $100$ & $4.46$ & $95.54$ & $350.78$ \\
		      \hline
		      $8$  & $100$ & $3.51$ & $96.49$ & $254.29$ \\
		      \hline
		      $9$  & $100$ & $2.54$ & $97.46$ & $156.83$ \\
		      \hline
		      $10$ & $100$ & $1.57$ & $98.43$ & $58.40$ \\
		      \hline
	      \end{tabular}
	\item [16.]
	      \begin{itemize}
		      \item [(a)] Scheme (i): $B_t=\dfrac{La_{\angl{n-t}i}}{a_{\angl{n}i}}=\dfrac{L(1-v_i^{n-t})}{1-v_i^n}$\\Scheme (ii): $B_t=\dfrac{La_{\angl{12(n-t)}i}}{a_{\angl{12n}i}}=\dfrac{L(1-v_j^{12(n-t)})}{1-v_j^{12n}}$\\Since $v_i^n=v_j^{12n}$, the above are equivalent.
		      \item [(b)] Scheme (i): $I_{(i)}=\dfrac{nLi}{1-v_i^n}$\\Scheme (ii): $I_{(ii)}=\dfrac{nLj}{1-v_j^{12n}}$\\Since $j=\frac{i^{(12)}}{12}$, $j<\frac{i}{12} \implies I_{(i)}>I_{(ii)}$
	      \end{itemize}
	\item [17.] $B_n=\frac{3}{4}L \implies a_{\angl{n}i}=\frac{3}{4}a_{\angl{2n}i} \implies v^n-1=\frac{3}{4}v^{2n}-\frac{3}{4} \implies v^n=\frac{1}{3} \implies \boxed{I_{n+1}=\frac{2}{3}K}$
	\item [26.]	$I_n=153.86 \implies iB_{n-1}=153.86 \implies i(X-P_{n-1})=153.86 \implies X=7240$\\$I_n=153.86 \implies K(1-v)=153.86 \implies K=1384.74$\\$P_1=K-I_1=K-iX=\boxed{479.74}$
	\item [29.]
	      \begin{itemize}
		      \item [(a)] $K_1v+K_2v^2+...K_nv^n=\frac{1+i}{1+i}+\frac{(1+i)^2}{(1+i)^2}+...+\frac{(1+i)^n}{(1+i)^n}=n$
		      \item [(b)] $B_t=K_{t+1}v+K_{t+2}v^2+...+K_nv^{n-t}=\frac{(1+i)^{t+1}}{(1+i)}+\frac{(1+i)^{t+2}}{(1+i)^2}+...+\frac{(1+i)^n}{(1+i)^{n-t}}=(n-t)(1+i)^t$
	      \end{itemize}
\end{itemize}
\end{document}