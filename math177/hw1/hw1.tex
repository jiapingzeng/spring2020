\documentclass{article}
\usepackage[margin=1in]{geometry}
\usepackage{setspace}
\usepackage{amsmath}
\usepackage{relsize}

\title{Math 177 Homework 1}
\date{4/8/2020}
\author{Jiaping Zeng}

\begin{document}
\setstretch{1.35}
\maketitle

\section*{Section 1.1}
\begin{itemize}
      \item [1.]
            After 1 year: $A(1)=10000*1.04=\boxed{10400}$\\
            After 2 years: $A(2)=10000*(1.04)^2=\boxed{10816}$\\
            After 3 years: $A(3)=10000*(1.04)^3=\boxed{11248.64}$\\
            Year 1 interest: $A(1)-A(0)=10400-10000=\boxed{400}$\\
            Year 2 interest: $A(2)-A(1)=10816-10400=\boxed{416}$\\
            Year 3 interest: $A(3)-A(2)=11248.64-10816=\boxed{432.64}$
      \item [3.]
            Balance after 12 months: $A(12)=10000*(1.01)^3*(1.0075)^{12-3}=\boxed{11019.70}$\\
            Average compound monthly interest: $11019.70=10000*(1+i)^{12} \implies \boxed{i=0.812\%}$
      \item [4.]
            Let the 10th year be the reference point; then $10000=10000*(1+4\%)^{10}-(1+5\%)*K*(1+4\%)^{10-4}-(1+5\%)*K*(1+4\%)^{10-5}-K*(1+4\%)^{10-6}-K*(1+4\%)^{10-7}$. Solving for $K$ results in $\boxed{K=979.93}$.
      \item [6.]
            Joe's accumulated amount (simple interest): $10*(1+0.11*10) + 30*(1+0.11*5)=67.5$\\
            Tina's accumulated amount (compound interest): $10*(1+0.0915)^n+30*(1+0.0915)^{2n}$\\
            We can solve for $n$ by setting the two amounts equal: $67.5=10*(1+0.0915)^n+30*(1+0.0915)^{2n}$ which results in $\boxed{n=3.364610}$.
      \item [11.]
            To simplify the calculations, we can look at the accumulation after $17*67=1139$ days.
            \begin{itemize}
                  \item [(a)]
                        17-day rate of $\frac{3}{4}\%$: $a(67)=(1+0.0075)^{67}=1.649752$\\
                        67-day rate of $3\%$: $a(17)=(1+0.03)^{17}=1.652848$\\
                        Therefore \boxed{\text{67-day rate of $6\%$}} results in more rapid growth.
                  \item [(b)]
                        17-day rate of $\frac{3}{2}\%$: $a(67)=(1+0.015)^{67}=2.711595$\\
                        67-day rate of $6\%$: $a(17)=(1+0.06)^{17}=2.692773$\\
                        Therefore \boxed{\text{17-day rate of $\frac{3}{2}\%$}} results in more rapid growth.
            \end{itemize}
      \item [12.]
            \begin{itemize}
                  \item [(a)]
                        Amount after 6 months: $1000*(1-9\%)*(1+25\%)*(1-1.5\%)=1120.4375$\\
                        6-month return: $(1120.4375-1000)/1000=\boxed{12.04\%}$
                  \item [(b)]
                        Amount after 6 months: $1000*(1-9\%)*(1-\frac{3.50}{4.00})*(1-1.5\%)=784.30625$\\
                        6-month return: $(784.30625-1000)/1000=\boxed{-21.57\%}$
            \end{itemize}
\end{itemize}

\section*{Section 1.2}
\begin{itemize}
      \item [2.]
            $v=\frac{1}{1+10\%}=0.909091$\\
            Child 1 (age 1): $PV_1=25000*v^{18-1}+100000*v^{21-1}=19810.47952$\\
            Child 2 (age 3): $PV_2=25000*v^{18-3}+100000*v^{21-3}=23970.68023$\\
            Child 3 (age 6): $PV_3=25000*v^{18-6}+100000*v^{21-6}=31904.97538$\\
            Total: $PV=PV_1+PV_2+PV_3=\boxed{75686.14}$
      \item [5.]
            Pick July 1, 2021 as time of reference and let payment needed then be $K$. Then, $200*(1+4\%)^{2021-2020}+300*v^{2022-2021}=100*(1+4\%)^{2021-2017}+K$. Solving for $K$ results in $\boxed{K=379.48}$.
      \item [7.]
            Let time 0 be the time of reference. Then, $100+200*v^n+300*v^{2n}=600*v^{10} \implies 100+200*0.75941+300*(0.75941)^2=600*(\frac{1}{1+i})^{10}$. Solving for $i$ results in $\boxed{i=0.03511}$.
      \item [8.]
            Let the cost of the machine in scenario $i$ be $M_i$.
            \begin{itemize}
                  \item [(a)] $M_1=20*(24000)*(1+\frac{0.75\%}{12})^{4*12}=\boxed{494613.54}$
                  \item [(b)] $M_2-200000*(\frac{1}{1+0.75\%})^{4*12}=M_1 \implies M_2=\boxed{634336.37}$
                  \item [(c)] $M_3=0.15*M_3*(\frac{1}{1+0.75\%})^{4*12}+M_1 \implies M_3=\boxed{552512.50}$
            \end{itemize}
      \item [12.]
            Let $i$ be the implied annual effective interest rate and the time of reference be 2 years from now. Then we get $1000*(1+i)^2+1092=2000*(1+i)$ which has \boxed{\text{no real solution}}.
      \item [16.]
            \begin{itemize}
                  \item [(a)] We can find $B_1$ by adding the initial balance ($B_0$) with interest and all transactions with their respective interests: $\boxed{B_1=B_0(1+i)+\sum_{k=1}^{n}a_k[1+i(1-t_k)]}$
                  \item [(b)] Let $0\leq p\leq n$, $t_0=0$ and $t_{n+1}=1$. Then, the balance after transaction $t_p$ can be represented by $B_p=B_0+\sum_{k=1}^{p}a_k$. To find the average balance, we can scale each $B_p$ by the time between $t_p$ and the next transaction, sum up all scaled $B_p$, then divide by the total time. This translates to \[\bar{B}=\dfrac{\sum_{p=0}^{n}B_p(t_{p+1}-t_p)}{1-0}\] By expanding the numerator, we can alternate the above to \[\bar{B}=\sum_{p=0}^{n}B_p t_{p+1}-\sum_{p=0}^{n}B_p t_{p}\] which is equivalent to \[\bar{B}=\sum_{p=1}^{n+1}t_p B_{p-1}-\sum_{p=0}^{n}t_{p}B_p\] By extracting the $t_{n+1}$ term of the first summation and the $t_0$ term of the second summation, we can then combine the two summations again \[\bar{B}=t_{n+1}B_n-t_0 B_0-\sum_{p=1}^{n}t_p (B_p-B_{p-1})\] Recall that $t_0=0$ and $t_{n+1}=1$; the first two terms of the right hand side simply reduce to $B_n$, which is equivalent to $B_0+\sum_{k=1}^{n}a_k$ by definition. In addition, $B_p-B_{p-1}=a_p$ by construction. We now have \[\bar{B}=B_0+\sum_{k=1}^{n}a_k-\sum_{p=1}^{n}t_p a_p\] By combining the remaining two summations, we get the final form \[\boxed{\bar{B}=B_0+\sum_{k=1}^{n}a_k(1-t_k)}\]
                  \item [(c)] We can verify the equation by substituting in $B_1$ and $\bar{B}$ from the previous parts: \[B_0(1+i)+\sum_{k=1}^{n}a_k[1+i(1-t_k)]=B_0+\sum_{k=1}^{n}a_k+[B_0+\sum_{k=1}^{n}a_k(1-t_k)]i\] We can see that they are indeed equal by reorganizing.
            \end{itemize}
      \item [17.]
            The original payment plan would mean paying on months 0 to 23, whereas the proposed one means playing on months 2 to 25. As a result, the savings is the difference between the present values of months 0 to 1 of the original plan and months 24 to 25 of the proposed plan. The savings then can be represented by $30+30*v-30*v^{24}-30*v^{25}=\boxed{12.68}$.
      \item [18.]
            Since the interests in the scenarios below are all simple interest, the annual interest rate of $10\%$ translates to a monthly rate of $\frac{10\%}{12}$ and a daily rate of $\frac{10\%}{365}$. In addition, the total amount deposited from January to March is $4*2500+3*1000=13000$.
            \begin{itemize}
                  \item [(a)]
                        \[
                              \text{Minimum monthly balance}=
                              \begin{cases}
                                    2500 & \text{Jan 1st - Jan 31st} \\
                                    6000 & \text{Feb 1st - Feb 28th} \\
                                    9500 & \text{Mar 1st - Mar 31st} \\
                              \end{cases}
                        \]
                        Therefore, total interest earned is $(2500+6000+9500)*\frac{10\%}{12}=150$. Then, the account balance is $13000+150=\boxed{13150}$.
                  \item [(b)]
                        \[
                              \text{Minimum daily balance}=
                              \begin{cases}
                                    2500  & \text{Jan 1st - Jan 15th}  \\
                                    3500  & \text{Jan 16th - Jan 31st} \\
                                    6000  & \text{Feb 1st - Feb 15th}  \\
                                    7000  & \text{Feb 16th - Feb 28th} \\
                                    9500  & \text{Mar 1st - Mar 15th}  \\
                                    10500 & \text{Mar 16th - Mar 31st}
                              \end{cases}
                        \]
                        The total interest earned in this case is $(2500*15+3500*16+6000*15+7000*13+9500*15+10500*16)*\frac{10\%}{365}=160.27$, which translates to an account balance of $13000+160.27=\boxed{13160.27}$.
                  \item [(c)]
                        Calculations for this scenario is similar to those of part (a), except that each minimum monthly balance should now include the interest earned from previous months. Therefore, we can calculate the balance month by month as below.\\
                        Balance on Jan 31st: $2500*(1+\frac{10\%}{12})+1000+2500=6020.83$\\
                        Balance on Feb 28th: $6020.83*(1+\frac{10\%}{12})+1000+2500=9571.01$\\
                        Balance on Mar 31st: $9571.01*(1+\frac{10\%}{12})+1000+2500=\boxed{13150.77}$
                  \item [(d)]
                        Similar to part (b) but with previous interest taken into account.\\
                        Balance on Jan 31st: $2500*15*\frac{10\%}{365}+3500*16*\frac{10\%}{365}+2500+3500=6025.62$\\
                        Balance on Feb 28th: $6025.62*15*\frac{10\%}{365}+7025.62*13*\frac{10\%}{365}+6025.62+3500=9575.41$\\
                        Balance on Mar 31st: $9575.41*15*\frac{10\%}{365}+10575.41*16*\frac{10\%}{365}+9575.41+3500=\boxed{13161.12}$
            \end{itemize}
\end{itemize}

\section*{Section 1.4}
\begin{itemize}
      \item [3.]
            We can convert both rates to their respective annual equivalent and form the following equation:\\
            $(1+\frac{0.15}{2})^2=(1+\frac{i^{(365)}}{365})^{365} \implies \boxed{i^{(365)}=0.144670}$.
      \item [4.]
            We can start by finding the balance of Eric at $t=7.5$ years: $A_{\text{Eric}}(7.5)=X*(1+\frac{i}{2})^{7.5*2}$. Then, the amount of interest Eric earns during the last 6 months of the 8th year would be $A_{\text{Eric}}(7.5)*\frac{i}{2}$. Mike, on the other hand, would earn $2X*i*\frac{1}{2}$ during the same period. We can solve for i by setting the two equal: $A_{\text{Eric}}(7.5)*i=2X*i*\frac{1}{2} \implies (1+\frac{i}{2})^{15}=2 \implies \boxed{i=0.0945882}$.
      \item [10.]
            We can solve this problem by equating the effective annual rates and trying different values of $m$:
            \begin{itemize}
                  \item [(a)] $1+18\%=(1+\frac{17\%}{m})^m \implies \boxed{m=4}$
                  \item [(b)] $1+18\%=(1+\frac{16\%}{m})^m \implies \boxed{\text{not possible}} \text{ even with } m=365$
            \end{itemize}
\end{itemize}

\section*{Section 1.5}
\begin{itemize}
      \item [1.]
            \begin{itemize}
                  \item [(a)] $4992*(1+8\%)^{\frac{1}{2}}=X \implies \boxed{X=5187.84}$
                  \item [(b)] $4992*(1+8\%*\frac{1}{2})=X \implies \boxed{X=5191.68}$
                  \item [(c)] $4992=X*(1-8\%)^{\frac{1}{2}} \implies \boxed{X=5204.52}$
                  \item [(d)] $4992=X*(1-8\%*\frac{1}{2}) \implies \boxed{X=5200}$
            \end{itemize}
      \item [2.]
            Since the discount rate is a simple discount, $A(t)=A(0)*\frac{1}{1-dt} \implies A(0)=A(t)*(1-dt)$. we can verify by substituting the given values: $99.941667=100*(1-\frac{28}{360}*0.750\%)$ which is indeed true. Similarly, we can verify the investment rate by verifying $A(t)=A(0)*(1+it) \implies 99.941667*(1+0.761*\frac{28}{360})=100$ which is also true.
      \item [4.]
            The percentage reduction can be found by dividing the price difference by the retail price: $\frac{30\%-15\%}{130\%}=\boxed{11.54\%}$ reduction.
      \item [5.]
            Bruce's balance after 10 years is $A_{\text{Bruce}}(10)=100*\frac{1}{(1-d)^{10}}$. Then, his interest during 11th year is $A_{\text{Bruce}}(10)*(\frac{1}{1-d}-1)=X$ (equation 1). Similarly, Robbie's balance after 16 years is $A_{\text{Robbie}}(16)=50*\frac{1}{(1-d)^{16}}$, thus resulting in an interest of $A_{\text{Robbie}}(16)*(\frac{1}{1-d}-1)=X$ (equation 2).\\
            We now have two equations with two unknowns ($d$, $X$). Solving numerically results in $d=0.109101$ and $\boxed{X=38.88}$.
      \item [11.]
            Let year 1 be the time of reference, then we can set up an equation as follows: $1000*(1+i)=1200*(1-d)$. Since $i=d$, we get $i=\frac{1}{11}=\boxed{0.0909091}$.
\end{itemize}

\end{document}